\documentclass[12pt]{article}
\usepackage{graphicx}
\usepackage{float}
%\usepackage{wrapfig}
\setlength{\textwidth}{6.5in}
\setlength{\textheight}{9.9in}
\setlength{\oddsidemargin}{0.0in}
\setlength{\evensidemargin}{0.0in}
\setlength{\topmargin}{-2.5 cm}
\setlength{\parskip}{0.7\baselineskip} % space between paragraphs
\setlength{\parindent}{0 cm} % increase this if you like paragraphs to be indented
\usepackage{hyperref}
\usepackage{xcolor}
\usepackage{natbib}  % For bibliography styles

\pagestyle{empty}

\newcommand{\e}{\mathrm{e}}

\begin{document}

\title{\bf{Lab 4: Q4}}
\author{Adam Omarali (1010132866)}
\maketitle

\textbf{4.1}

a)
\begin{figure}[H]
    \centering
    \includegraphics[width=0.6\textwidth]{images/4.1.png}
\end{figure}
b) The ODE is unstable and a source.

c) The eigenvalues are $\lambda_1 = \frac{5 + \sqrt 5}{2}$ and $\lambda_2 = \frac{5 - \sqrt 5}{2}$. This corresponds to $\lambda_1 > \lambda_2 > 0$, which is an unstable source.

\textbf{4.2}

a)
\begin{figure}[H]
    \centering
    \includegraphics[width=0.6\textwidth]{images/4.2.png}
\end{figure}

b) The ODE is stable and a sink.

c) The eigenvalues are $\lambda_1 = \frac{-5 + \sqrt 5}{2}$ and $\lambda_2 = \frac{-5 - \sqrt 5}{2}$. This corresponds to $\lambda_1 < \lambda_2 < 0$, which is a stable sink.

\textbf{4.3}

a)
\begin{figure}[H]
    \centering
    \includegraphics[width=0.6\textwidth]{images/4.3.png}
\end{figure}

b) The ODE is an unstable saddle point.

c) The eigenvalues are $\lambda_1 = 2$ and $\lambda_2 = -1$. This corresponds to $\lambda_1 > 0 > \lambda_2$, which is an unstable saddle point.

\textbf{4.4}

a)\begin{figure}[H]
    \centering
    \includegraphics[width=0.6\textwidth]{images/4.4.png}
\end{figure}

b) The ODE is an unstable saddle point.

c) The eigenvalues are $\lambda_1 = 1$ and $\lambda_2 = -2$. This corresponds to $\lambda_1 > 0 > \lambda_2$, which is an unstable saddle point.

\textbf{4.5}

a)\begin{figure}[H]
    \centering
    \includegraphics[width=0.6\textwidth]{images/4.5.png}
\end{figure}

b) The ODE is a stable spiral point.

c) The eigenvalues are $\lambda_1 = \lambda_2 = -2$. This corresponds to $\lambda_1 = \lambda_2 < 0$, which is a stable spiral point.

\textbf{4.6}

a) \begin{figure}[H]
    \centering
    \includegraphics[width=0.6\textwidth]{images/4.6.png}
\end{figure}

b) The ODE is an unstable spiral point.

c) The eigenvalues are $\lambda_1 = \lambda_2 = 2$. This corresponds to $\lambda_1 = \lambda_2 > 0$, which is an unstable spiral point.

\textbf{4.7}

a) \begin{figure}[H]
    \centering
    \includegraphics[width=0.6\textwidth]{images/4.7.png}
\end{figure}

b) The ODE is a stable center.

c) The eigenvalues are $\lambda_1 = 2i$ and $\lambda_2 = -2i$. This corresponds to $\lambda_1 = -\lambda_2$, which is a stable center.

\textbf{4.8}

a) \begin{figure}[H]
    \centering
    \includegraphics[width=0.6\textwidth]{images/4.8.png}
\end{figure}

b) The ODE is an stable center.

c) The eigenvalues are $\lambda_1 = 2i$ and $\lambda_2 = -2i$. This corresponds to $\lambda_1 = -\lambda_2$, which is an stable center.

\textbf{4.9}

a) \begin{figure}[H]
    \centering
    \includegraphics[width=0.6\textwidth]{images/4.9.png}
\end{figure}

b) The ODE is a stable center.

c) The eigenvalues are $\lambda_1 = i$ and $\lambda_2 = -i$. This corresponds to $\lambda_1 = -\lambda_2$, which is a stable center.

\textbf{4.10}

a) \begin{figure}[H]
    \centering
    \includegraphics[width=0.6\textwidth]{images/4.10.png}
\end{figure}

b) The ODE is an stable center.

c) The eigenvalues are $\lambda_1 = i$ and $\lambda_2 = -i$. This corresponds to $\lambda_1 = -\lambda_2$, which is a stable center.
\end{document}