\documentclass[12pt]{article}
\usepackage{graphicx}
\usepackage{float}
%\usepackage{wrapfig}
\setlength{\textwidth}{6.5in}
\setlength{\textheight}{9.9in}
\setlength{\oddsidemargin}{0.0in}
\setlength{\evensidemargin}{0.0in}
\setlength{\topmargin}{-2.5 cm}
\setlength{\parskip}{0.7\baselineskip} % space between paragraphs
\setlength{\parindent}{0 cm} % increase this if you like paragraphs to be indented
\usepackage{hyperref}
\usepackage{xcolor}
\usepackage{natbib}  % For bibliography styles

\pagestyle{empty}

\newcommand{\e}{\mathrm{e}}

\begin{document}

\title{\bf{ODE Lab 5}}
\author{Adam Omarali (1010132866)}
\maketitle

\section*{Q1}
\textbf{a)}
\begin{figure}[H]
    \centering
    \includegraphics[width=0.9\textwidth]{images/q1.png}
\end{figure}

\textbf{b)}
All solutions (100\%) decay with oscillation.

\textbf{c)}
The solution to this ODE is $y(t) = e^{-\frac{1}{2}t} \left( c_1 \cos 2t + c_2 \sin 2t \right)$.
This makes sense since the $e^{-\frac{1}{2}t}$ component decays as $t \rightarrow \infty$ and the $\cos 2t$ and $\sin 2t$ components oscillate.

\section{Q2}
\textbf{a)}
\begin{figure}[H]
    \centering
    \includegraphics[width=0.9\textwidth]{images/q2.png}
\end{figure}

\textbf{b)}
All solution (100\%) grow.

\textbf{c)}
The solution to this ODE is $y(t) = c_1e^{(\frac{-\sqrt{3}}{2} + 1)t} + c_2e^{(\frac{-\sqrt{3}}{2} - 1)t}$. This makes sense since as $t \rightarrow \infty$, $c_2e^{(\frac{-\sqrt{3}}{2} - 1)t} \rightarrow 0$ and $c_1e^{(\frac{-\sqrt{3}}{2} + 1)t} \rightarrow \pm \infty$ depending on the sign of $c_1$.


\section*{Q3}
\textbf{a)}
\begin{figure}[H]
    \centering
    \includegraphics[width=0.9\textwidth]{images/q3.png}
\end{figure}

\textbf{b)}
All solutions (100\%) decay.

\textbf{c)}
The solution to this ODE is $y(t) = c_1e^{\frac{-\sqrt{3}+\sqrt{2}}{2}t} + c_2e^{\frac{-\sqrt{3}-\sqrt{2}}{2}t}$. This makes sense since both terms approach 0 as $t \rightarrow \infty$.

\section*{Q4}
\textbf{a)}
Solving for the roots of characteristic equation gives the general solution $y(t) = c_1e^{-t}cos 2t + c_2e^{-t}sin 2t + c_3cos t + c_4sin t$.

\textbf{b)}
Both of the exponential terms will decay to zero as $t \rightarrow \infty$. The $\cos 2t$ and $\sin 2t$ terms will oscillate indefinitely. Therefore, the solution decays while oscillationing, but never approaches 0.
Therefore, no solutions (0\%) decay, grow, decay with oscillation or grow with oscillation. Nearly all solutions (100\%) oscillate indefinitely aside from initial conditions where $c_3 = c_4 = 0$.

\section*{Q5}
(a) $0 < r_1 < r_2$. The solution will grow.

(b) $r_1 < 0 < r_2$. The solution will grow.

(c) $r_1 < r_2 < 0$. The solution will decay.

(d)$ r_1 = \alpha + \beta i, \  r_2 = \alpha - \beta i \ |\  \alpha < 0$. The solution will decay with oscillation.

(e) $r_1 = \alpha + \beta i, \  r_2 = \alpha - \beta i \ |\  \alpha = 0$. The solution will oscillate indefinitely.

(f) $r_1 = \alpha + \beta i, \  r_2 = \alpha - \beta i \ |\  \alpha > 0$. The solution will grow with oscillation.

\section*{Q7}

\begin{figure}[H]
    \centering
    \includegraphics[width=0.9\textwidth]{images/q7.png}
\end{figure}

The solution osciallates accurately but begins to diverge from the iode
solution more and more as t increases. This could be improved with a smaller step size (currently h = 0.1).
\end{document}